This study has shown that deep generative models can be applied to generate synthetic data sets that can be used to boost the performance of existing discriminative models. The experiments revealed some settings where this likely is the case. It was also illustrated that with a simple task and a large data set, a discriminative model can be used to evaluate the quality of the generated data sets. Although this study focuses on data augmentation, this finding may well have a bearing on how to benchmark different \acrshort{gans} in general. 

The scope of this study was limited in terms of time and computational resources, and therefore several questions still remain to be answered. Notwithstanding these limitations, this study strenghtens the idea that synthetic data sets can be used as drop-in replacements for existing data sets and it lays the groundwork for future research into generating usable data with deep generative models.

\section{Future work}
%TODO: Write stuff here if I want to.

% Explore more models

% Compare different discriminative models for the same generative model

% Test if models trained for one purpose generate datasets applicable for another purpose: pupil detection -> gaze estimation
